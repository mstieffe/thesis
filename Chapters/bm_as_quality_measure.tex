% Chapter Template

\chapter{Backmapping as a Quality Measure for Coarse-Grained Models} % Main chapter title

\label{bm_as_quality_measure} % Change X to a consecutive number; for referencing this chapter elsewhere, use \ref{ChapterX}

\section{Motivation and Problem Description}

\section{Coarse-Grained Model}

\section{Results and Discussion}

\begin{figure}
  \centering
      \includegraphics[width=0.8\textwidth]{./Figures/quality_of_cg_models/tmbt_cg.pdf}
  \caption{TMBT CG}
  \label{FIG:TET_morphgibbs}
\end{figure}

\begin{figure}
  \centering
      \includegraphics[width=0.8\textwidth]{./Figures/quality_of_cg_models/tmbt_bm_only1.pdf}
  \caption{TMBT CG}
  \label{FIG:TET_morphgibbs}
\end{figure}

\begin{figure}
  \centering
      \includegraphics[width=0.8\textwidth]{./Figures/quality_of_cg_models/tmbt_forces.pdf}
  \caption{TMBT CG}
  \label{FIG:TET_morphgibbs}
\end{figure}

\begin{table}
\begin{tabular}{ c|c|c } 
 & DBM & EM \\
\hline
reference & 0.0056 & 0.0423 \\ 
model 1 & 0.0064 &  0.0868 \\ 
model 2 & 0.0063 &  0.0866 \\ 
model 3 & 0.0064 0.0884 &  
\caption{ RMSD CG - CG(BM)}
\label{TAB:sPS_chem_trans}
\end{tabular}
\end{table}
