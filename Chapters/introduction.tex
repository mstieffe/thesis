\chapter{Introduction} 
\label{introduction} 


Scientific modeling has always been an indispensable part of natural science. Modeling is the construction of a simplified representation for a phenomena in order to describe a certain aspect of the world.


Modeling/physics/ancient greeks/heaven geometry, 

physical theories span scales from thermodynamics to statistical mechanics, universe to quantum, different scales of models, 
thermodynamic, statistical mechanics

molecular processes, scales of molecular processes

computer, numerical model, MD simulations

multiscale modeling, soft matter, coarse-graining
BACKMAPPING

Machine learning, generative modeling, deep learning, high dimensional data

Thesis, intersection MS and ML, backmapping, DBM, applications



A model denotes a simplified representation of certain aspect of the world.

physics, modeling, different scales


Molecular processes are fundamental to life, and for their general understanding it is of
importance to understand their structural and dynamical properties. The involved—sometimes
intracellular—components can be resolved via microscopy techniques such as STORM with
a resolution of 20 nm to 50 nm [1], cryo-EM with a resolution of 3 Å to 15 Å [2, 3], or X-ray
crystallography with a resolution of < 1 nm (which was first used to resolve a protein structure
in the 1950s [4]).

Soft condensed matter, like polymers or complex liquids, is characterized by interaction energies of
the order of kBT at room temperature T , where kB is the Boltzmann constant.1–4 Thus, thermal fluctu-
ations can induce structural and conformational changes in soft materials, which makes these systems
highly flexible. As a consequence, it can take several seconds for soft materials to reach an equilibrium
state at macroscopic length scales (millimeter to meter). This makes it difficult to study soft matter with
the aid of computer simulations. But, computer simulations enable to study soft matter at resolutions
difficult to access with common experimental techniques.5–9 The problem with computer simulations
arises from the fact that the common methods, classical molecular dynamics (MD)10 and Monte-Carlo
(MC)11 simulations, are limited to shorter length and time scales12 than those required to account for
the equilibration of soft matter at macroscopic length scales. Therefore, there is a necessity to reach
larger length and time scales with computer simulations on the one hand. On the other hand, one has
to simultaneously account for length and time scales at which microscopic changes occur (picometer,
femtoseconds), for example the formation of hydrogen bonds. Hence, modeling of soft materials is a
multiscale problem,13 which is illustrated in figure 1

.............This thesis lies at the intersection of com-
putational healthcare and machine learn-
ing. The field of machine learning has
seen enormous development over the last
several decades. Advances in deep learn-
ing (LeCun et al. , 2015), powered by
Graphical Processing Units (GPUs), en-
able practitioners to build supervised ma-
chine learning algorithms which make pre-
dictions from high-dimensional data using
millions of datapoints. We have begun
to see visible successes of machine learn-
ing in domains such as computer vision
(Krizhevsky et al. , 2012), natural lan-
guage processing (NLP) (Mikolov et al.
, 2013b) and neural machine translation
(Bahdanau et al. , 2014)...........



