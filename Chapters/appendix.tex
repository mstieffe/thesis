% Chapter Template

\chapter{Appendix} % Main chapter title
\label{appendix} % Change X to a consecutive number; for referencing this chapter elsewhere, use \ref{ChapterX}

\begin{figure}[ht]
 \includegraphics[width=1.0\textwidth]{./Figures/appendix/bm_sps/architecture.pdf}
\caption{CNN architecture with residual connections of the generator (left) and critic (right). The first part of the generator consists of an encoder which learns a lower dimensional embedding of the condition given by $\epsilon_i ( \mathbf{x})$ using several residual connections and one pooling layer. Noise $z$ and the atom type $\mathbf{c}_i$ are concatenated to this low dimensional embedding and is fed into the decoding part of the generator, which again consists of several residual connections and an upsampling layer. The critic learns a one dimensional embedding of the condition $\epsilon_i ( \mathbf{x})$ and the target/fake atom $\gamma_i$/$\hat{\gamma}_i$ using residual layers and a final dense layer. Throughout the whole architecture layernorm is applied for regularization and LeakyRelus are used as nonlinearities.}
\label{APPENDIX:DBM_architecture}
\end{figure} 
 
 
%\begin{figure*}
%    \centering
%   \begin{subfigure}[b]{0.475\textwidth}
%        \centering
%        \includegraphics[width=\textwidth]{./Figures/appendix/bm_sps/adv_c.pdf}
%        \caption{Adversarial loss for the critic}    
%        \label{adv_loss_c}
%    \end{subfigure}
%    \hfill
%    \begin{subfigure}[b]{0.475\textwidth}  
%        \centering 
%        \includegraphics[width=\textwidth]{./Figures/appendix/bm_sps/adv_g.pdf}
%        \caption{Adversarial loss for the generator}       
%        \label{adv_loss_g}
%    \end{subfigure}
%    \vskip\baselineskip
%    \begin{subfigure}[b]{0.475\textwidth}   
%        \centering 
%        \includegraphics[width=\textwidth]{./Figures/appendix/bm_sps/energy_loss.pdf}
%        \caption{Forcefield based loss}    
%        \label{energy_loss}
%    \end{subfigure}
%    \hfill
%    \begin{subfigure}[b]{0.475\textwidth}   
%        \centering 
%        \includegraphics[width=\textwidth]{./Figures/appendix/bm_sps/com_loss.pdf}
%        \caption{Center of mass loss}       
%        \label{com_loss}
%    \end{subfigure}
%    \caption{ Different loss terms during training for the training set (red) and the validation set (blue)}
%    \label{APPENDIX:DBM_sps_losses}
%\end{figure*}


\begin{figure}[htbp]
	\begin{center}
		\includegraphics[width=0.6\linewidth]{./Figures/appendix/bm_sps/sm_p2_and_no_p.pdf}
		\caption{Low-dimensional representation of the local environments of sPS monomers at $T = 568$~K. For each panel, snapshots are backmapped from identical CG configurations. Landmarks of reference structures (grey) and projections of structures generated with chemically-specific (red) and chemically-transferred (blue) models trained with (a) $\mathcal{C}^{(2)}_{\text{pot}}$ (b) without regularization. Reprinted from \cite{stieffenhofer2021adversarial}. }
		\label{APPENDIX:sm_p2_and_no_p}
	\end{center}
\end{figure}

\begin{figure}
  \centering
      \includegraphics[width=0.8\textwidth]{./Figures/quality_of_cg_models/tmbt_bm.pdf}
  \caption{Pair correlation functions $g(r)$ for the AA reference system, backmapped in-distribution test set and backmapped test set for all CG models. Results obtained with DBM (left) and EM scheme (right) are displayed, including non-bonded (a)-(b) C-C, (c)-(d) C-N and (e)-(f) N-N correlations.}
  \label{BMQM:atomistic_rdf_all}
\end{figure}
